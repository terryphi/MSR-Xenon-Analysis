\subsection{Valid Theory}
The validity of existing work may be framed along a spectrum from totally invalid to totally valid. A totally invalid theory would be characterized by major phenomena missing, whereas a totally valid theory would encapsulate all the phenomena and provide predictive capabilities. It is assumed that the theory of xenon migration in molten salt reactors, as described in the ORNL reports, and articulated by Shimazu is essentially correct in its formulation and application.  That being said, the theory in itself has yet to be fully validated.  The a priori xenon model developed in ORNL-4069 was felt to be essentially correct and used in online xenon calculations in the MSRE.\cite[p. 16] {ORNLTM3464} ORNL-TM-3464 reports,
\begin{quote}
    “Although we achieved reasonable success in describing the steady-state xenon poisoning with both helium and argon cover gas, we could no adequately describe the transient behavior”. \cite[p. 91]{ORNLTM3464}
\end{quote}
The authors believe that there is mixed evidence to support the assumption of a valid theory, and choose to refrain from further commentary. 