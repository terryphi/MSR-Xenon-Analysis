\subsection{Negligable Adsorption}
In prior xenon models, adsorption onto reactor surfaces has been assumed to be negligible.  Xenon can either be adsorped directly or a xenon proginator can be adsorped.  Our review of literature has revealed four potential adsorption sinks in MSRs: the graphite pore-space, corrosion products, circulating particulate, and reactor structural material. 

ORNL-4069 states xenon is not significantly adsorbed on graphite at MSRE operational temperatures (~930 K). \cite[p. 42]{ORNL4069}  Justification for this claim is given by reference to two journal articles, one by Salzano and Eshaya, and another by Cannon et al.  The report by The article by Salzano and Eshaya concerns the sorption\footnote{sorption refers to the superordinate process which encapsulates both adsorption and absorption.} of xenon on graphite at high temperature, and concludes,
\begin{quote}
    "Above 500 \degree C the quantity of xenon held by surface adsorption is probably less than 1 \% [of]  that held in the voids." \cite{Salzano1962}
\end{quote}
No information on the negligable adsorption assumption was found in the paper by Cannon et al. \cite{Cannon1962}

ORNL-TM-3464 reported corrosion scales can contain a significant holdup of iodine. \cite[p. 5]{ORNLTM3464} Reference is made to ORNL-TM-228, presumably as justification. \cite{Burch1962}   The evidence for iodine adsorption in ORNL-TM-228  is manifest with the statement,
\begin{quote}
"A model which postulated that a psedu-equilibrium existed between iodine in solution and that adsorbed on the walls fit most of the data obtained when it assumed that only 10 \% of the iodine in the high pressure system was circulating, [and] the other 90 \% [was] being adsorbed on the walls." [ibid. p.10]
\end{quote}
That being said, no explicit mention of corrosion scale was found in ORNL-TM-228.

The article