\subsection{Iodine and Tellerium Behavior}
The behavior of I-135 and Te-135 is governed through an \textit{in-solution assumption} wherein they are assumed to remain in solution in the fuel-salt and not migrate to the bubbles, circulating voids or cover-gas.  

The iodine in-solution assumption may not be entirely accurate. In ORNL-4865 it was found that salt samples from the MSRE had a portion\footnote{Most samples had between 30-60\% of their I-131 missing} of their expected I-131 missing. \cite[p. 28]{ORNL4865} Contrary to this, ORNL-4069 claims very little iodine was found in the MSRE off-gas system.\cite[p. 42]{ORNL4069} Furthermore, ORNL-4069 states an unpublished internal report on the volitization of free iodine in the MSRE supports the iodine in-solution assumption vis-\'a-vis its thermodynamic properties. [ibid.]  Furthermore, when the \textit{iodine in-solution assumption} was stated in ORNL-TM-3464, A reference to ORNL-3913  was given, presumably as justification for the assumption. \cite[pp. 38-40] {ORNL3913} ORNL-3913 describes an experiment wherein iodine was removed from a FLiBe melt through HF.  No explanation was found as to how the findings of the experiment relate to the validity of the in-solution assumption


The tellerium in-solution assumption is justified by considering the 29 s half-life of Te-135.\cite[p. 42]{ORNL4069}   ORNL-4865 claims Tellerium would act as a dissolved gas in a manner similar to xenon given its vaporization temperature. \cite[p. 29]{ORNL4865}  We posit a mass transfer processes affecting tellerium would need to do so on a time-scale comparable to its half-life in order for the in-solution approximation to be invalid.

Finally, the burn-out of I-135 and Te-135 is assumed to be negligible.  This is justified on the grounds that the neutron absorption cross-section of both Te-135 and I-135 is on the order of 1b.  \cite{TENDL15}