\subsection{Dynamics of Constants}
In prior analyses, the mass transfer coefficients, diffusion coefficients, and Henry's constants used in MSR xenon analyses are assumed to be invariant with respect to dynamic behavior.  That is, they are essentially expressible in terms of an expression which includes no time derivatives.  ORNL-4069 derived mass transfer coefficients from a heat/mass analogue of the Dittus Bottler equation. \cite[p. 71]{ORNL4069}
 

Liu and Fukuda measured the Nusselt number of a horizontal heated cylinder in a helium gas flow. \cite{Liu02} The measurement was performed with both transient and steady-state heat production in the cylinder. In the transient case,  The heat applied followed an exponential function in time.  Liu and Fukuda found the ratio of transient and steady-state Nusselt number exceeded 1 on when the period of the heat transient was shorter than 1 s. 

Mass transfer analogies provide a formulaic conversion between heat transfer coefficients and mass transfer coefficients.  Given there exists a particular case wherein the transient Nusselt number was different than the steady-state Nusselt number, it follows the transient mass transfer coefficients in an MSR may be different than a steady-state mass transfer coefficients.  No work using transient mass transfer coefficients or any other transient constant of proportionality was found.