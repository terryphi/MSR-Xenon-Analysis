\subsection{Nobel Gas Decay and Graphite Non-Homogenity} \label{sec:nobel}

Prior investigations of MSR xenon behavior have assumed both the mass diffusion coefficient of the graphite, $D_g$ and the graphite void fraction, $\epsilon$, were invariant with respect to position and time; that is, the graphite stringers were assumed to be homogeneous right cylinders. Furthermore, the diffusion of xenon into the graphite is assumed to be uniform with respect to the radial and axial location of the surface into which the xenon diffuses  Converse to this, we have identified three phenomena that may lead to stringer non-homogeneity:  Noble gas decay,  pressure differential induced xenon cross-flow, and  physical non-homogeneity,

In terms of noble gas decay, when Xe-135, a noble gas, undergoes beta minus decay, it transmutes into the alkali metal, Cs-135.  If Cs-C or another compound forms accumulates within the pore space, the void fraction and mass diffusion coeffecient will be affected. The formation of Cs-C from Xe-135 was investigated by Baes.\cite[p. 15]{ORNL4037} Baes used some assumed values to calculate partial pressure of Cs-135 in the graphite due to migration from the partial pressure of Xe-135 in the salt-graphite surface Baes concluded,
    \begin{enumerate}
        \item The amount of Cs-135 accumulation in the graphite is ‘very small’.
        \item If it is assumed that all the Cs-135 born in the graphite were to remain in the graphite, the rate of accumulation would be low.
    \end{enumerate}
No investigations on the accumulation of other species due to noble gas or other fission product transmutations were found.



In addition to the operational causes of non-homogeneity in the graphite stringers, the graphite stringers may form non-homogeneity in the manufacturing process.  ORNL-4148 reports the MSRE graphite stringers were observed to be non-homogeneous. \cite[p.25 ]{ORNL4148} Furthermore, it was noted that impregnation treatments, through which the graphite pore size is reduced, can create non-homogeneity in the graphite structures.