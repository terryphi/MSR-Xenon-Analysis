\subsection{Xenon Behavior}
Xenon is a noble gas,  and does not normally form chemical species. We assume xenon is in a free atomic, gaseous state.  This may not be entirely true.  Xenon was found to form a compound with fluorine, XeF2, under conditions similar to those found in MSRs.  \cite{Tram06} A discussion of noble gas compounds is given by Asimov. \cite[p. 228]{Asimov79} Mackenzie and Wiswall irradiated a sample of xenon and fluorine with gamma radiation from a Co-60 source and observed the synthesis of xenon compounds. \cite{MacKenzie1963}  

That being said, it is foreseeable that the formation of xenon compounds would depend on the availability of free fluorine to bond with.  Details about the amount of free fluorine in MSRE fuel salt were not found. The following statements may be useful in a discussion about the existence of free fluorine.  Sridharan and Allen state that the higher valence states of UF\_n compounds added to the fuel salt can undergo multiple reductions, which would liberate free fluorine into the salt.  \cite[p. 252]{Lantelme13} Ignatiev states polyvalent salts, such as UF\_(n), are \textit{acidic} and tend to form complexes with F-.  \cite[p. 266]{Gaune12}  Delpech reports the BeF2 in FLiBe tends to form compounds with flourides; the ammount of free fluorine in FLiBe depends on the LiF/BeF2 ratio; and, in the case of 66-34 mol\% LiF-BeF2, the activity of free fluoride atoms is very low. 
\cite[p. 39]{Delpech2010} If the activity of fluorine in a salt melt is high, then it is foreseeable that there may be some binding of xenon.  Nevertheless, for our purposes, we  assume that xenon does not form any compounds in the MSR.   No investigations nor mass transfer theories were found that accounted for the potential of xenon compound formation.

