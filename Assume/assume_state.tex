\subsection{The State of Mater}

 The melting and boiling temperatures of the three elements are shown in table \ref{boilMeltTbl}.\begin{table}[h]
\begin{tabularx}{\linewidth}{ |X|X|X| }
  \hline
 \textbf{Chemical Species} & \textbf{Melting Point [K]} & \textbf{Boiling Point [K]} \\
  \hline 
  Tellerium  & 723  & 1261   \\
  \hline
  Iodine & 387 & 458 \\
   \hline
   Xenon & 161  &  165 \\
   \hline
\end{tabularx}
\caption{Melting and boiling points of the poison species \cite[p. 4-118]{CRCChemPhy97}}
\label{boilMeltTbl}
\end{table}  
 
There is a difference in phase transition temperatures, with respect to their parrent chemical species, for some isotopes.  Case in point is deutrium which has a boiling point about three degrees higher than protium. \cite[p. 23]{Goodwin2003} The phase transition temperature shown in \ref{boilMeltTbl} is for the chemical species, which is mixture of the isotopes of the particular species.  The difference between a particular poison isotope's phase transition temperature and that of its parent chemical species can be assumed to be negligible.  This is justified given the difference in mas between the poison isotope the atomic mass of its corresponding element is less than one percent.  Given the MSRE outlet temperature of 936 K, we see all the poison isotopes are either in a liquid or gaseous state.  No information was found that indicated either way that there is a change in MSR xenon behavior as the temperature of an MSR transitioned across a particular phase transition temperature. In prior work, iodine and tellerium were assumed to dissolve in the fuel salt whereas the xenon was assumed to come out of solution and effectivley form a mixtures. It is unclear what effect this has on phase transition temperatures of the contituates and if distillation of the poison species occurs.