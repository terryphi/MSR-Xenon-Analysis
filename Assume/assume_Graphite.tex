\subsection{Graphite Stringer Considerations}

The work of ORNL-4069 and Shimazu both modeled graphite stringers as a cylindrical transient diffusion initial boundary value problem,

%\begin{align}
%     C_t &= \frac{D_g}{\epsilon}(C_{rr} + \frac{1}{r}C_r) - (\sigma \phi + \lambda_{Xe})C,\\
%    C_r(r=0) &= 0m \\
%    C_r(r=R) &= k_m (C_FS - \frac{HRT}{\epsilon}C)
%    . \text{\cite[p.45, p.807]{ORNL4069,Shimazu1977}}   
%\end{align}

\begin{subequations}
\label{eqn:cyl}
\begin{align}
    &C = C(r,t) \\ 
    &C_t = \frac{D_g}{\epsilon} \left (C_{rr} + \frac{1}{r}C_r \right ) - (\sigma \phi + \lambda_{Xe})C \label{eqn:cylRD}\\        
        &C_r(r=0) = 0 \\
        &C_r(r=R) = k_m \left (C_{FS} - \frac{HRT}{\epsilon}C \right ) \\
        &C(t=0) = 0
\end{align}
\end{subequations}
Abstracting the cylindrical Laplacian into a differential operator, we can write equation \ref{eqn:cylRD} as a reaction-diffusion equation,
\begin{equation}
    C_t = \frac{D_g}{\epsilon} \nabla^2C - (\sigma \phi + \lambda_{Xe})C.
\end{equation}

As shown in ORNL-TM-0728, the graphite stringers are square-cylinders with rounded square fuel channels cut into their sides.  \cite[p. 80]{Robertson65}  Therefore, application of equation \ref{eqn:cyl} thereby transforms the concave geometry of the graphite stringer into an \textit{equivalent cylinder} as illustrated in figure \ref{img:GSXfrm}.
\begin{figure}[ht]
\centering
\includesvg[width=0.5\textwidth]{stringerTransform.svg}
\caption{Transformation of Graphite Stringer Geometry}
\label{img:GSXfrm} 
\end{figure}

No justification nor validation was found for this transformation.  Glicksman and Lienhard state approximation of geometry by cylinders is a \textit{tactic} used in solving heat transfer problems. \cite[p. 73]{Glicksman2016}  By analogy, if complex geometry can be simplified to cylindrical geometry for heat transfer, it follow, by analogy, that the same holds true for mass transfer.

Equation \ref{eqn:cyl} further assumes the porous-media advection term is negligible. Diffusion experiments on MSRE graphite specimen, however, show that gas flow across the specimen was induced when a pressure differential was applied across it.\cite[p. 26]{ORNL4148} Each fuel channel operates at a particular pressure which is a function of the fuel salt flow velocity and the position of the fuel-channel in the reactor.\cite[p. 14]{ORNLTM378}  Two fuel-channels operating at different pressures and contacting the same graphite stringer will induce a pressure gradient across the stringer.  Nothing has been found which investigates the effects of xenon advection across graphite stringers.

Equation \ref{eqn:cylRD} assumes the graphite pore space to be entirely interconnected, and xenon at any point in the graphite is assumed to be able to migrate to any other point in the graphite.  As expounded by Bear, it is possible for a porus medium to be comprised of multiple disjoint domains. \cite[p. 7]{Bear2012} Furthermore, as exposited in section \ref{sec:nobel}, there is the potential for strucutrual non-homogenity to exist within the MSRE graphite. This would cause $D_g$ or $\epsilon$ to vary as a function of position.  No Investigations were found into these potential variations.