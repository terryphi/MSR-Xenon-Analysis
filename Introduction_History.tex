\subsection{Xenon In Historical MSRs}
There were three experimental MSRs that were built and operated: the Aircraft Reactor Experiment (ARE); the Pratt and Whitney Aircraft Reactor-1 (PWAR-1); and, the Molten Salt Reactor Experiment (MSRE).

The Aircraft Reactor Experiment (ARE) was designed to demonstrate basic features in an effort to develop a nuclear propulsion source on an aircraft. \footnote{A reactor referred to as the \textit{Fused Salt Reactor Experiment} (FSRE) is described in the report by Ergen.\cite{Egen1957}  The description of this reactor is similar to that of the ARE.  We posit that the FSRE is the same reactor as the ARE.}  ARE used a NaF-ZrF4-UF4 fuel salt, a beryllium oxide moderator, and outlet temperature of 1133 K.\cite[p. 62]{OSullivan2015} The ARE operated for 100 MW-hours over the course of nine days in 1954. [ibid] Bettis reports that the positive temperature coefficient of reactivity related to xenon poisoning in solid fuel reactors was considered to be significant enough to warrant abandoning them in design considerations for the ARE project.\cite{Bettis1957} The ARE used a xenon removal system built into the fuel-pumps. The xenon removal system was comprised of a mixing chamber in which the fuel salt was sprayed through a helium atmosphere on to the mixing chamber wall.  The result was a foam/salt mixture with a high gas-interface surface area.  The large surface area facilitates the transfer the gaseous fission products from the salt to the gas. The foam/salt mixture was then pumped into an expansion volume, where the helium and gaseous products therein were removed via an off-gas system.\cite[pp. 31-32]{Fraas1956} An experiment performed on the ARE, named Experiment H-11,   showed that no more than 5\% of the xenon in the produced by the reactor remained in the fuel salt.\cite[p. 67]{ORNL1845}  This poison fraction was determined by finding the predicted reactivity lost from xenon poisoning without any xenon removal, and comparing it to measured amount of xenon poisoning. 

The Pratt and Whitney Aircraft Reactor (PWAR-1) was zero-power test bed with a beryllium moderator and reflector. \cite[p. 286]{Dolan2017} Critical, zero-power experiments with external heating were performed on the reactor in 1957.  \cite[p. 2]{ORNL2536} No information on xenon behavior in the PWAR-1 was found. 

The Molten Salt Reactor Experiment (MSRE) was an 8 MWth graphite moderated MSR built and operated in the 1960s.\cite[p. 376]{MacPherson1985}  The MSRE underwent two distinct operational phases, one with a U-235 based fuel salt, starting in 1965, and another with a U-233 based fuel salt, starting in 1968.\cite[p. 6]{Forsberg2006} The two major reports that detail xenon behavior in the MSRE were ORNL-4069 and ORNL-TM-3464.\cite{ORNL4069,ORNLTM3464}

Two models were generated in ORNL-4069, one with, and another without bubbles included. The experience leading to the decision to include bubbles in the model is described in ORNL-TM-1796. \cite[p. 14]{ORNLTM1796} Both of these models included considerations for the xenon stripper,  absorption into graphite, and xenon progenitors.  The reactor was subdivided into 72 annular regions, and the xenon poisoning was formulated for each region.  The per-region poisoning was weighted by the adjoint flux, and the poisoning for the entire reactor was found by summing across all the regions.  Sensitivity analyses were performed with both models. ORNL-4069 (published in 1967) indicated that at higher void fractions, the majority of the xenon would be found in the circulating bubbles rather than the graphite or fuel salt. \cite[p. 56]{ORNL4069} This finding is preceded by a report in ORNL-4037, wherein Prince et al. state,

\begin{quote}
"This plot [showing xenon distribution in graphite, fuel-salt, and bubbles as a function of circulating void fraction] shows show circulating bubbles work to decrease the loss of reactivty to $^{135}Xe$ .  As the void fraction is increased, most of the dissolved xenon migrates to the bubbles, dropping the dissolved xenon concentration greatly. The concentration potential [driving force] necessary for $^{135}Xe$  to migrate to the graphite is reduced accordingly." \cite[p. 15]{ORNL4037}
\end{quote}

This is a new development, for in 1961, the analysis by Miller, which ignored circulating bubbles, showed that given his assumptions, the majority of xenon was found in the graphite. \cite[p. 7]{Miller1961} The claim in ORNL-4069, that bubbles may contain significantly more xenon than the fuel salt, was corroborated, in ORNL-TM-3027 (published 1970 by Engel et al.) which states that a few bubbles dispersed in the fuel salt can contain far more xenon than all of the salt.\cite[p. 47]{ORNLTM3027} ORNL-4069 concludes with the following xenon-related conclusions\cite[p. 58]{ORNL4069}:

\begin{enumerate}
  \item The presence or lack of bubbles has a significant impact on the outcome of the model.  When bubbles were included in the model, the xenon poison fraction could be made to agree with preliminary observed xenon poison fractions.
  \item Many assumptions including the volatization of iodine were made in the model, and the model ought not be considered final.
  \item $^{135}Xe$ poisoning shows an insensitivity to the void fraction and diffusion coefficient of the graphite. 
\end{enumerate}

The other major report detailing xenon in the MSRE was ORNL-TM-3464, written in 1971. \cite{ORNLTM3464} The report provides a description of processes affecting $^{135}Xe$ behavior in the MSRE; predictions about MSRE xenon behavior; observations of xenon behavior during the MSRE operation; an analysis of xenon within the MSRE cover-gas; details the generation of a model that describes MSRE xenon behavior; and, discusses the results of this model. 

Two separate xenon models were developed in ORNL-TM-3464, one for soluble, and another for insoluble cover-gas.  The model that assumed a soluble cover-gas used a separate, coupled cover-gas model, whereas the model that assumed an insoluble cover-gas used an integrated cover-gas model.  The structure of the xenon model was as follows: regions contained sub-regions, which contained nodes.  One isotopic species concentration was associated with each node. The fuel-loop was treated in four regions, each treated as a well-stirred tank.  The regions were the pump-bowl piping and heat exchanger,  reactor core, and piping to the pump bowl
.  The piping and heat exchanger was further divided into liquid and gas-bubble sub-regions.  The reactor core was subdivided into a liquid, gas-bubble sub-regions, and six graphite sub-regions.  The piping to the pump bowl was identical to the piping and heat exchanger sub-region in that it contained both liquid and gas-bubble sub-regions.  Finally, the pump bowl had sub-regions for the gas-space, old bubbles, and liquid.  Each sub-region had nodes for $^{135}I$, $^{135}Xe$, and $^{135m}Xe$ where appropriate. The graphite within the core was subdivided into four radial regions with mass-transfer between the regions accounted for by diffusion.  Subdivision was in the radial direction, however, unlike the xenon model in ORNL-4069, no facility was made for breakdown in the axial direction. Each radial region had appropriate regional flux and nuclear importance incorporated into its behavior.  Each of the nodes had a material balance equation written on it in the form of a system of first-order linear differential equations. This system was then solved for both the steady-state solution vector as well as for the transient behavior.

ORNL-TM-3464 concludes the following\cite[96]{ORNLTM3464}:
\begin{enumerate}
    \item The totality of xenon behavior for MSRs had not been accurately predicted by prior analyses
    \item Subsequent analyses have been partially successful in describing xenon behavior, but there continues to be areas of uncertainty
    \item There was a considerable difference in xenon behavior depending on if Helium or Argon cover gas was used.
    \item Attempts at reactor behavior description required liquid/gas mass transfer coefficients, and stripping efficiencies substantially different than the predicted values.
    \item Description of xenon behavior required considerations for mass transfer from the circulating voids to the graphite. 
    \item The circulating void to graphite behavior was a function of both circulating void size and fraction.
\end{enumerate}
Finally, ORNL-TM-3464 posses a set of five research questions. The authors of ORNL-TM-3464 believe the answer to these research question will lead to increased accuracy in the modeling of MSR xenon behavior.  No detailed attempts at answering these questions have been found, but some ...TODO

circ voidds/ bubbles fraction, seze.
xenon behavior...