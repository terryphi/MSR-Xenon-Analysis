HELLO

\section{A Lexicon of Graphtie Stringers}

During our discussions of graphite stringers, we noticed the lexography of graphite stringers was sparse in accepted terminology to discuss xenon behavior. Furthermore, no articulated definitions of existing graphite stringer terminology was found We therefore now propose a grammar of graphite stringers through which discussions may be held.  An MSR core, as viewed from the top, cut along the mid-plane, has two major components, the peripheral region, and the lattice block; illustrated in figure \ref{img:regions}.
\begin{figure}[ht]
\centering
\includesvg[width=\textwidth]{Regions.svg}
\caption{MSR Regions}
\label{img:regions} 
\end{figure}

 The lattice region is comprised of a number of graphite stringers assembled in a lattice.  Each lattice position is occupied by a unit cell, which consists of a graphite stringer and one or more fuel channels that have been cut into the graphite stringer.  There is upward salt flow within the fuel channels of the lattice region and the peripheral region.  Thus, MSRs are characterized by a number of fuel channels and graphite stringers, and the number of fuel channels is prescribed by the number, geometry, and arrangement of the graphite stringers. The lattice region may be further subdivided into a number of radial flow regions based on the Reynolds number of the flow in the fuel channels in that region.  The hydraulic model in ORNL-TM-0378 uses five regions to describe the MSRE core. \cite[P. 13]{ORNLTM378}
 
Figure \ref{img:stringer_types} shows four potential unit cells and their arrangement into a lattice.
\begin{figure}[ht]
\centering
\includesvg[width=\textwidth]{StringerTypes.svg}
\caption{Types of MSR Stringers}
\label{img:stringer_types} 
\end{figure}

The unit cell consists of a graphite stringer body and a number of fuel channels slots; illustrated in figure \ref{img:unit_cell}. When two or more unit cells are arranged in a lattice, their meshed configuration creates a number of fuel channels.  In this particular instance, each side of the unit cell contains only one half of a fuel channel; however, this is not necessarily the rule. In Figure \ref{img:stringer_types}, column B, for example, each unit cell contains a full fuel channel. The perimeter of the stringer body is subdivided into a wetted perimeter and a dry perimeter.

\begin{figure}[ht]
\centering
\includesvg[width=\textwidth]{UnitCell.svg}
\caption{A Unit Cell}
\label{img:unit_cell} 
\end{figure}

Furthermore, there is an inter-stringer space between graphite stringers arranged in a lattice, shown in figure \ref{img:interstringer_space}. Either the graphite stringer are in direct contact with each other, or there is a substance between them. There may also be regions along the length of the stringer that are in direct contact and other regions with substance between the stringers.  The characteristics of the inter-stringer space may also evolve as the reactor ages (creep expansion, irradiation, fission product evolution and migration, thermal expansion, etc). No information nor modeling efforts were found about this inter-stringer space. 
\begin{figure}[ht]
\centering
\includesvg[width=\textwidth]{interstringer_space.svg}
\caption{Detail of Interstringer Space}
\label{img:interstringer_space} 
\end{figure}