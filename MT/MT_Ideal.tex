\subsection{Ideal Gasses, Dillute Solutions, and Deviation Therefrom}
Fletcher reports a macroscopic discrete phase inclusion will form if the radius excedes the a embryo of dispersed phase atoms exceeds the critical radius,
\begin{equation}
    r_c = \frac{2\sigma}{\Delta G_V}.
\end{equation}\cite{Fletcher1958}
Westh and Haynes investigated the bounds of the domain of concentrations where Henry's Law is applicable,  the Henry region, in water and hexane using a number of solutes. \cite{Westh1998} Westh and Haynes were unable to determine the Henry region due to the limits of the calorimeter used; however, they were able to give a lower limit for non-Henry behavior on the order of $10^{-4}$, for hexane solvents and solutes, and $10^{-5}$, for a water solute and a number of solvents.   As previously mentioned, Grimes, Blander and Watson established solubility limits for xenon in molten salts, above which no further xenon could dissolve in their experimental apparatus. \cite{Grimes1958,Blander1959,Watson1962}

\begin{figure}[ht]
\centering
\includesvg[width=0.5\textwidth]{XenonConc.svg}
\caption{MSR Gas Concentration Map}
\label{concMap} 
\end{figure}
Given these observations, we posit the following: Consider figure \ref{concMap}.  This figure represents the phase space of xenon and other-gas concentrations within the molten salt system.  There is then, three regions, H,N,and B.  The Henry region, H, is where Henry's law is applicable; The Non-Henery region, N is where Henry's law is not applicable, but the salt melt has yet to reach gas saturation; finally, B, the bubble-out region, is where the salt melt has reached saturation and additional gas evolution forms dispersed gas-phase inclusions rather than dissolving into the salt melt.  No evidence has been found that indicates the H/N boundary exhibits any sort of sharp discontinuous behavior, and we suspect the transition from Henry-like to non-Henry-like behavior exhibits a \textit{smooth} character.  One potential criterion for the H/N boundary is the xenon concentration such that the ratio of the xenon to partial pressure is greater than multiple, K, of the Henry constant,
\begin{equation}
    C_i^L \; S.T. \; \frac{C_i^L}{p_i^g} > KH_i.
\end{equation}
Since the Henry constant is a function of temperature, pressure, and composition, it likewise follows the location of the  H/N boundary is a function of temperature, pressure, and composition. 
Consider a volume of salt melt with xenon and other gasses dissolved in it. This salt melt also contains actinides that are undergoing fission.  As the actinides undergo fission, a fraction of their fission products are gaseous.  If we track the behavior of these gaseous fission products, including xenon, then there is a certain probability per unit time that the gaseous products will join a gas embryo of radius greater than the critical radius.  This probability is a function of the quantity of gas within the salt melt.  We posit the N/B boundary is the total gas concentration such that a gas atom is more likely than not to join a gas embryo of radius greater than critical radius.  Since the critical radius is a function of the surface tension and Gibbs free energy, and these are functions of temperature and composition, and for the case of the Gibbs free energy, the pressure, the N/B boundary is likewise a function of temperature, pressure and composition.

All the xenon modeling efforts to date have focused on modeling efforts within the H region.  No work was found that did not use Henry's law the formulation of the mass transfer equations.